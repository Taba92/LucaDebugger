\documentclass[a4paper,12pt]{article}

\begin{document}
Questo è un esempio di comunicazione client-server,//
in cui \textit{N client fanno richieste ad 1 server,che per ogni richiesta spawna 1 worker}.\\
Nel caso un worker fallisse la gestione della richiesta,il server stesso risponderà con un errore al client.\\
I cerchi neri sono processi con flag=true,mentre quelli bianchi con flag=false:

\setlength{\unitlength}{1cm}
\begin{picture}(8,8)(0,0)
\put(0,7){\vector(1,0){1}}
\put(1,7){= RICHIESTA}
\put(0,6){\line(1,0){1}}
\put(1,6){= SPAWN WORKER}
\put(7,7){\oval(2,1)}
\put(8,7){= WORKER}
\put(1,5){\circle{1}}
\put(1,4){.....}
\put(1,3){\circle{1}}
\put(1.5,5){\vector(2,-1){2}}
\put(1.5,3){\vector(2,1){2}}
\put(3.7,4){\circle*{1}}
\put(4,4){\line(2,1){2}}
\put(4,4){\line(2,-1){2}}
\put(7,5){\oval(2,1)}
\put(7,3){\oval(2,1)}
\end{picture}
\end{document}
